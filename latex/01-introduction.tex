\documentclass[latex/main.tex]{subfiles}

\begin{document}
\section{Introduction}

\subsection{Scientific Motivation} 
The brain functions over a wide range of timescales, or time autocorrelation decay rates, from rapid synaptic events to low frequency activity that synchronize neuron populations and larger-scale networks of regions \cite{buzsaki_large-scale_2004}. The diverse timescales over which neural processes operate are spatially organized, forming maps that provide mechanistic insights into how different regions integrate and process information. Multimodal evidence suggests that timescales are an essential part of brain organization, shaped by increasing temporal receptive fields moving from sensory to associative cortical regions \cite{raut_hierarchical_2020, gao_neuronal_2020, hasson_hierarchy_2008}. These temporal receptive fields are analogous to the spatial receptive fields of sensory brain areas, which define the region of space to which they respond \cite{hubel_receptive_1959}. These results suggest that hierarchical timescales may emerge from properties intrinsic to brain functional anatomy.\\

These timescale gradients align with the well-established functional hierarchy in the brain, where sensory areas, which process rapidly changing stimuli, exhibit faster timescales than higher-order association areas involved in more complex cognitive processes that unfold over longer durations \cite{murray_hierarchy_2014, hasson_hierarchy_2008, stephens_place_2013}. The emergence of hierarchical timescales likely depends on multiple mechanisms. Computational modeling by Li and Wang (2022)*\cite{li_hierarchical_2022} suggest three key factors: 1) a brain-wide gradient in the strength of synaptic excitation, 2) differences in the electrophysiological properties of excitatory versus inhibitory neurons, and 3) the balance between excitatory inputs from distant brain regions and inhibitory inputs from local circuits.\\

Further, timescales are not static but can be dynamically modulated by experimental manipulations. Pharmacological agents, such a propofol and serotonergic drugs, can alter intrinsic timescales, affecting the temporal integration of information in the brain \cite{huang_timescales_2018, shinn_functional_2023}. Further, research has shown changes in timescales across behavioral states, such as sleep deprivation and wakefulness, as well as neuropsychiatric disorders like autism and schizophrenia \cite{meisel_decline_2017, watanabe_atypical_2019, wengler_distinct_2020}. These findings demonstrate that timescales have broad relevance to both structural and functional properties of the brain.\\

Seminal research on timescales primarily focused on non-human animals, particularly macaques, using electrophysiological recordings of neural activity with high temporal resolution \cite{murray_hierarchy_2014, cirillo_neural_2018, nougaret_intrinsic_2021, manea_intrinsic_2022}. Yet, because of sampling limitations of electrode arrays, only sparse sets of neurons (usual from cortical regions) can be recorded from. With sparse data it is difficult to infer on the spatial organization of timescale maps. This instead requires brain-wide coverage, which is not feasible with invasive electrophysiology.\\

In the present paper, we focus on resting functional MRI (rfMRI), which measures spontaneous fluctuations of the blood oxygen level-dependent (BOLD) signal in the absence of external stimuli, and provides noninvasive full-brain coverage of hemodynamic processes at frequencies $<0.1$ Hz \cite{raut_hierarchical_2020, he_scale-free_2011}. The BOLD signal does not directly measure neural activity but rather the hemodynamic changes that are linked to the underlying electrophysiological signal \cite{logothetis_what_2008}--it reflects a mix of both electrophysiological and metabolic processes and a means to investigate high-spatial resolution timescale maps. Several studies have found that rfMRI timescale maps consistently align with those from different imaging modalities (EEG, ECoG, MEG) across human and animal models \cite{raut_hierarchical_2020, shafiei_topographic_2020, lurie_cortical_2024, mitra_lag_2014, kaneoke_variance_2012, wengler_distinct_2020, shinn_functional_2023, manea_intrinsic_2022, ito_cortical_2020, muller_core_2020}.\\

\subsection{Methodological Motivation}
For an overview, neural timescales are generally calculated in the 1) autocorrelation domain, 2) time domain, or 3) frequency domain. In the \textbf{autocorrelation domain}, as introduced by Murray (2014), the timescale is defined by an exponential decay fit of the autocorrelation function (acf) \cite{murray_hierarchy_2014, rossi-pool_invariant_2021, cirillo_neural_2018, ito_cortical_2020, runyan_distinct_2017, zeraati_flexible_2022, nougaret_intrinsic_2021, wasmuht_intrinsic_2018, muller_core_2020, maisson_choice-relevant_2021, li_hierarchical_2022, shafiei_topographic_2020}. Similar methods exist to characterize area under the acf \cite{wengler_distinct_2020, manea_intrinsic_2022, watanabe_atypical_2019} or the time lag where the acf reaches a threshold \cite{zilio_are_2021, raut_hierarchical_2020, golesorkhi_temporal_2021}. Alternatively, it is possible to fit timescales directly in the \textbf{time domain} with a first order autoregressive model (ar-1) \cite{kaneoke_variance_2012, meisel_decline_2017, huang_timescales_2018, lurie_cortical_2024, shinn_functional_2023, shafiei_topographic_2020}; which closely reflects the exponential decay model \cite{murray_hierarchy_2014} but is computationally more efficient and scalable to high-dimensional applications like rfMRI. Further, ar-1 models have higher test-retest reliability than exponential decay models for rfMRI datasets \cite{huang_timescales_2018, lurie_cortical_2024}. Lastly, the \textbf{frequency domain} approach was developed to account for neural oscillations, as timescales are properties of the aperiodic component of the signal and can be biased when the process has periodic or oscillatory components \cite{donoghue_parameterizing_2020, gao_neuronal_2020}. These components can be estimated and removed in the frequency domain, and this approach involves fitting the residual power spectral density (psd) shape with a Lorentzian function or measuring low frequency power to infer timescales \cite{gao_neuronal_2020, zeraati_flexible_2022, fallon_timescales_2020}. Given previous research showing that rfMRI is aperiodic and scale-free, characterized by 1/f-like spectral properties \cite{he_scale-free_2011, he_temporal_2010}, the present paper focuses only on the autocorrelation and time domain methods -- specifically the exponential decay and autoregressive models.\\ 

\subsubsection*{Problem Statement}
From this overview it is clear that a primary challenge for timescales research in the lack of standardized definitions and estimation methods across studies. Different research groups use varied approaches to define and calculate timescales, leading to inconsistent results and interpretations. Additionally, many parameterization methods assume specific stochastic processes that might not capture the complexities of neural data, potentially biasing timescale estimates and (more often) their standard errors. While some solutions for robust timescale estimators under general conditions exist \cite{zeraati_flexible_2022, donoghue_parameterizing_2020, gao_neuronal_2020}, there is currently no method to robustly estimate standard errors. Typically, existing papers report only point estimates without quantifying the uncertainty of these estimates, hindering inference, comparison across brain regions, and generalization across subjects or groups.\\

\subsubsection*{Proposed Solution}
Focusing on the exponential decay model (autocorrelation domain) and autoregressive model (time domain) we introduce robust estimators for the variance of timescale parameters under general assumptions, and establish their theoretical properties (bias, consistency, limiting variance). This enables the assignment of standard errors to traditional point estimates, allowing for scientific inference on timescale maps of the brain. To validate theoretical approaches, we use simulations and rfMRI from the Human Connectome Project (HCP) dataset. Simulations demonstrate that the proposed methods achieve nominal bias in estimating timescale and standard error maps. Further, application to the HCP dataset allow us to interpret estimated timescale maps within the context of established findings to ensure results are consistent with the extent literature on neural timescales.\\

\end{document}