\documentclass[latex/main.tex]{subfiles}

\begin{document}
\section{Introduction}

\subsection{\textit{Scientific Motivation}} 
Brain activity spans a range of timescales, from the rapid dynamics of synaptic events to the slower oscillatory activities that coordinate neuron populations and large-scale networks \citep{buzsaki_large-scale_2004}. These timescales are spatially organized and align with anatomical maps, progressively increasing from sensory to associative brain regions \citep{raut_hierarchical_2020, gao_neuronal_2020, hasson_hierarchy_2008}. Multimodal evidence suggests that these timescales arise from intrinsic properties of brain anatomy and reveal how different regions integrate and process information over time.\\

Timescale maps align with the well-established functional hierarchy in the brain, where sensory areas, which process rapidly changing stimuli, exhibit faster timescales than higher-order association areas involved in complex cognitive processes that unfold over longer durations \citep{murray_hierarchy_2014, hasson_hierarchy_2008, stephens_place_2013}. This timescale gradient is spatially associated with myelination levels and patterns of gene expression, as demonstrated by studies utilizing human electrophysiology, MEG, and microarray-based gene expression profiling \citep{gao_neuronal_2020, shafiei_neurophysiological_2023}. The emergence of hierarchical timescales likely depends on multiple other mechanisms. Computational modeling by \citet{li_hierarchical_2022} suggests three key factors: 1) a brain-wide gradient in synaptic excitation strength, 2) differences in the electrophysiological properties of excitatory neurons versus inhibitory neurons, and 3) the balance between excitatory inputs from distant brain regions and inhibitory inputs from local circuits.\\

In addition to these intrinsic mechanisms, there is growing evidence that neuronal timescales are dynamic and modulated by experimental manipulations or behavioral demands. Pharmacological agents, such a propofol and serotonergic drugs, can alter intrinsic timescales, affecting the temporal integration of information in the brain \citep{huang_timescales_2018, shinn_functional_2023}. Further, research has shown changes in timescales with development, across sleep deprivation and wakefulness, neuropsychiatric disorders like autism and schizophrenia, as well as naturalistic behaviors \citep{martin-burgos_development_2024, meisel_decline_2017, watanabe_atypical_2019, wengler_distinct_2020, manea_neural_2024}. Taken together, these findings demonstrate that timescales have broad relevance to both structural and functional properties of the brain.\\

Seminal research on timescales has primarily focused on non-human animals, particularly macaques, using electrophysiological recordings of neural activity with high temporal resolution \citep{murray_hierarchy_2014, cirillo_neural_2018, nougaret_intrinsic_2021, manea_intrinsic_2022}. Yet, because of sampling limitations of electrode arrays, only sparse sets of neurons, usual from cortical regions, can be recorded from. With sparse sampling in space it is difficult to infer on the large-scale spatial organization of timescale maps. This instead requires brain-wide coverage, which is not feasible with invasive electrophysiology.\\

In the present paper, we focus on resting functional MRI (rfMRI), which measures spontaneous fluctuations of the blood oxygen level-dependent (BOLD) signal in the absence of external stimuli, and provides noninvasive full-brain coverage of hemodynamic processes at frequencies $<0.1$ Hz \citep{raut_hierarchical_2020, he_scale-free_2011}. Compared to techniques like EEG, ECoG, and MEG, which offer high temporal resolution but relatively coarse spatial resolution, rfMRI offers dense sampling in space but sparse sampling in time. While the BOLD signal does not directly measure neural activity, it reflects hemodynamic changes associated with underlying electrophysiological signals \citep{logothetis_what_2008}, providing a means to investigate high-spatial resolution timescale maps. Several studies have found that rfMRI timescale maps consistently align spatially with those from other imaging modalities across human and animal models \citep{raut_hierarchical_2020, shafiei_topographic_2020, lurie_cortical_2024, mitra_lag_2014, kaneoke_variance_2012, wengler_distinct_2020, shinn_functional_2023, manea_intrinsic_2022, ito_cortical_2020, muller_core_2020}. \\

\subsection{\textit{Methodological Motivation}}
Neural timescales are generally estimated in the 1) time domain, 2) autocorrelation domain, or 3) frequency domain. The most common is the \textbf{autocorrelation domain}, as introduced by \citet{murray_hierarchy_2014}, where the timescale is defined by a nonlinear exponential decay fit of the sample autocorrelation function (ACF) \citep{rossi-pool_invariant_2021, cirillo_neural_2018, ito_cortical_2020, runyan_distinct_2017, zeraati_flexible_2022, nougaret_intrinsic_2021, wasmuht_intrinsic_2018, muller_core_2020, maisson_choice-relevant_2021, li_hierarchical_2022, shafiei_topographic_2020}. Similar methods exist to characterize area under the ACF \citep{wengler_distinct_2020, manea_intrinsic_2022, watanabe_atypical_2019} or the time lag where the ACF crosses a predefined threshold \citep{zilio_are_2021, raut_hierarchical_2020, golesorkhi_temporal_2021}. Alternatively, it is possible to fit timescales directly in the \textbf{time domain} with a linear first order autoregressive model (AR1) \citep{kaneoke_variance_2012, meisel_decline_2017, huang_timescales_2018, lurie_cortical_2024, shinn_functional_2023, shafiei_topographic_2020}; which closely reflects the exponential decay model \citep{murray_hierarchy_2014} but is computationally more efficient and therefore scalable to high-dimensional applications like rfMRI. Further, time domain linear models have higher test-retest reliability than autocorrelation domain nonlinear models for rfMRI datasets \citep{huang_timescales_2018, lurie_cortical_2024}. Lastly, the \textbf{frequency domain} approach was developed to account for neural oscillations, as timescales are properties of the aperiodic component of the signal and can be biased when the process has periodic or oscillatory components \citep{donoghue_parameterizing_2020, gao_neuronal_2020}. These components can be more easily estimated and removed in the frequency domain, and this approach involves fitting the residual power spectral density (PSD) shape with a nonlinear Lorentzian function or measuring low frequency power to infer timescales \citep{gao_neuronal_2020, manea_neural_2024, zeraati_flexible_2022, fallon_timescales_2020}. Given previous research showing that rfMRI is aperiodic and scale-free, characterized by 1/f-like spectral properties \citep{he_scale-free_2011, he_temporal_2010}, the present paper focuses on the time and autocorrelation domain methods and the respective linear and nonlinear timescale models.\\ 

\subsection{\textit{Problem Statement and Proposed Solution}}
From this overview it is clear that a primary challenge for timescales research in the lack of standardized model definitions and estimation methods across studies. Different research groups use varied approaches to define and calculate timescales, leading to inconsistent results and interpretations. Additionally, many parameterization methods rely on restrictive assumptions, potentially biasing timescale estimates and (more often) their standard errors. While some solutions exist for robust timescale estimators under general conditions \citep{zeraati_flexible_2022, donoghue_parameterizing_2020, gao_neuronal_2020}, there is currently no method to robustly estimate the corresponding standard errors. Typically, existing papers report only point estimates without quantifying the uncertainty of these estimates, hindering inference, comparison across brain regions, and generalization across subjects or groups.\\

Focusing on the time domain linear model and autocorrelation domain nonlinear model we introduce robust estimators for the variance of timescale parameters under general assumptions, and establish their theoretical properties (bias, consistency, limiting variance). This enables the assignment of standard errors to traditional point estimates, allowing for scientific inference on timescale maps of the brain. To validate theoretical approaches, we use simulations and rfMRI from the Human Connectome Project (HCP) \citep{van_essen_wu-minn_2013}. Simulations demonstrate that the proposed methods achieve nominal bias in estimating timescale and standard error maps. Further, application to the HCP dataset allow us to interpret estimated timescale maps within the context of established findings to ensure results are consistent with the extent literature on neural timescale organization.\\

\end{document}