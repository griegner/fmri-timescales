\documentclass[latex/main.tex]{subfiles}

\begin{document}
\section{Discussion}

This study addresses gaps in current methodology for estimating neural timescales by introducing robust statistical techniques for mapping timescales across brain regions. Providing a detailed account of the large-sample properties of the two most commonly applied timescale estimators -- the time-domain linear model and the autocorrelation-domain nonlinear model -- we show that both behave consistently under broad assumptions of stationarity and ergodicity. Importantly, we demonstrate that the standard errors of both estimators are also consistent under the same general conditions, enabling statistical inference and hypothesis testing. This advancement addresses a notable limitation in most neural timescale studies, which often report only point estimates without appropriate measures of uncertainty, thereby limiting inference and testing.\\

Through parameter recovery simulations, we demonstrate that both models yield valid timescale estimates along with corresponding standard errors as a way to asses precision of estimation. Comparing the two, the time-domain linear model \ref{sec:time-domain-linear-model} offers greater computational efficiency and estimation stability for small timescales. In contrast, the autocorrelation-domain nonlinear model \ref{sec:autocorrelation-domain-nonlinear-model} provides greater estimation stability for large timescales, i.e., when long-range temporal dependencies are present. Across both models, the introduction of Newey-West corrected standard errors mitigates the downward bias in the standard error estimators under a broad range of time processes, enhancing the reliability of inference and testing \citep{newey_simple_1987}. This is shown in AR(2) and empirical resting fMRI (rfMRI) simulations, where the fitted models are misspecified relative to the data-generating process. Lastly, when applied to rfMRI data from the Human Connectome Project, both methods produce timescale and t-statistic maps that appear consistent with previously reported anatomical hierarchies in the brain \citep{van_essen_wu-minn_2013}. Timescale cortical maps from a range of other studies converge with the present rfMRI results to show that timescales are larger in associative versus sensory processing cortices \citep{raut_hierarchical_2020, shafiei_topographic_2020, lurie_cortical_2024, mitra_lag_2014, kaneoke_variance_2012, wengler_distinct_2020, shinn_functional_2023, manea_intrinsic_2022, ito_cortical_2020, muller_core_2020}. This provides some assurance that the methods described here are measuring biologically interpretable parameters about the "memory" or "persistence" of neural processes. That said, the interpretation of timescale maps is beyond the scope of the present paper and a topic for future research.\\

This segues to potential limitations in our approach. Although the use of rfMRI offers dense spatial sampling relative to other imaging modalities (EEG, MEG, ECoG), its sparse temporal sampling may introduce bias and variance in estimation. The estimator properties we discussed (consistency and limiting variance) depend on large-sample asymptotics. In practice, we use finite samples, and when the underlying hemodynamic process is sampled at a low frequency and exhibits strong temporal dependence, the effective sample size may be too small \citep{afyouni_effective_2019, kaneoke_variance_2012}. Furthermore, the non-specific metabolic and neuronal origins of the hemodynamic signal make mechanistic timescale interpretations difficult \citep{raut_hierarchical_2020, he_scale-free_2011}. On the methodological side, while our models handle misspecification to some extent, cases remain where the underlying process may deviate from the assumptions of stationarity and ergodicity, impacting the reliability of timescale estimates \citep[chapter~14.7]{hansen_econometrics_2022}. Extending model definitions to account for nonstationarity, where autocorrelations are time-dependent, might provide more accurate maps, especially in task-based or dynamic fMRI paradigms \citep{he_scale-free_2011}. Additionally, adding standard errors to the frequency-domain approach to timescale estimation would allow for the direct modeling of oscillations, which is important when working with electrophysiological recordings of the brain \citep{donoghue_parameterizing_2020, gao_neuronal_2020}.\\

In conclusion, this paper introduces robust estimators for timescales and their standard errors in rfMRI, enabling more rigorous statistical comparisons across brain regions, experimental conditions, and subjects. This advances the field toward more reliable and interpretable neural timescale maps. The methods presented here for mapping the spatial organization of timescales allow for inference and testing by moving beyond point estimates to incorporate variability. The work lays the methodological foundation for future research on the role of timescales in brain structure and function.
\end{document}