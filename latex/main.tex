\documentclass[9pt]{article}

\usepackage{subfiles} % include subfiles
\usepackage{fullpage} % 1-in margins
\usepackage{amsmath, amsfonts, amssymb} % equations
\usepackage{graphicx} % figures
\usepackage{float} % figures
\usepackage{hyperref} % hyperlinks
\usepackage[square,numbers]{natbib} % citations + references
\bibliographystyle{abbrvnat} % citations + references
\usepackage{fontspec} % font
\usepackage[defaultsans]{opensans} % font
\setmainfont[BoldFont={Open Sans Semibold}]{Open Sans} % bold font

\title{Estimating fMRI Timescale Maps}
\date{}

\begin{document}
\maketitle

\section*{Abstract}
Timescales describe the rate at which time-dependent processes decay, encompassing the activity of individual neurons, neural populations, and large-scale brain networks. The spatial organization of timescales across brain regions offers mechanistic insights into how these regions integrate and process information over time. Despite their functional significance, current methods for estimating timescales often rely on restrictive assumptions and report only point estimates, limiting statistical inference and comparisons across individual brains or experimental conditions. This paper introduces two robust models for timescale estimation, a time-domain linear model and an autocorrelation-domain nonlinear model, both evaluated under general assumptions. We establish the theoretical properties of these models (bias, consistency, and limiting variance) enabling the assignment of standard errors to point estimates, which facilitates hypothesis testing and the construction of confidence intervals. Sampling bias and variance are also evaluated through simulations and timescale maps are estimated using resting fMRI from the Human Connectome Project. Comparatively, the linear model is more computationally efficient and scalable to mass-univariate settings like fMRI, while the nonlinear model is more sensitive to long-range temporal dependencies. Both models employ robust standard errors to address biases under model misspecification, improving reliability in both simulated and empirical settings. Taken together, these methods yield cortical timescale maps consistent with known anatomical hierarchies, enhancing our understanding of neural dynamics. This work provides a robust framework for mapping and inferring neural timescales, facilitating more reliable comparisons across brain regions, experimental conditions, and individuals.

\subfile{latex/01-introduction}
\subfile{latex/02-methods}
\subfile{latex/03-results}
\subfile{latex/04-discussion}

\section{Code and Data Availability}
All simulation results and fMRI timescale maps, inclusive of the code by which they were derived, can be accessed on \href{https://github.com/griegner/fmri-timescales}{GitHub/fmri-timescales}. The code is under the open-source MIT license, permitting access and reuse with attribution. The Human Connectome Project's young adult dataset (ages 22-35; 2018 release) used in this study is publicly accessible under a \href{https://www.humanconnectome.org/storage/app/media/data_use_terms/DataUseTerms-HCP-Open-Access-26Apr2013.pdf}{data usage agreement}, which outlines specific terms for data usage and sharing.

% references
\bibliography{latex/zotero}


\end{document}