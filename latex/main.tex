\documentclass[9pt]{article}

\usepackage{subfiles} % include subfiles
\usepackage{fullpage} % 1-in margins
\usepackage{amsmath, amsfonts, amssymb} % equations
\usepackage{graphicx} % figures
\usepackage{hyperref} % hyperlinks
\usepackage[style=nature]{biblatex} % citations + references
\usepackage{fontspec} % font
\usepackage[defaultsans]{opensans} % font
\setmainfont[BoldFont={Open Sans Semibold}]{Open Sans} % bold font
\addbibresource{latex/zotero.bib} % zotero bibliography

\title{Estimating fMRI Timescale Maps}
\author{Gabriel Riegner}

\begin{document}
\maketitle

\section{Abstract}
The brain functions over a wide range of timescales, from rapid synaptic events to low frequency activities that synchronize neuron populations and larger-scale networks of regions. The diverse timescales over which neural processes operate are spatially organized, forming maps that provide mechanistic insights into how different regions integrate and process information. Despite their scientific importance, current parameterization methods are inconsistently defined across studies and make restrictive assumptions about the underlying stochastic process. Generally, existing research only reports point estimates without standard errors,  limiting the ability to draw inferences, compare brain regions, and generalize findings across subjects or groups. To address this gap, we propose both linear and nonlinear timescale models that are robust under general assumptions and scalable to high-dimensional fMRI datasets. The statistical properties of these estimators -- bias, consistency, and limiting variance -- are established, enabling the assignment of standard errors to traditional point estimates, and thus facilitating hypothesis testing and the construction of confidence intervals. Simulations and resting-state fMRI analyses demonstrate that these estimators enable valid scientific inferences on timescale maps of the brain.

\subfile{latex/01-introduction}
\subfile{latex/02-methods}
\subfile{latex/03-results}
\subfile{latex/04-discussion}
\subfile{latex/05-appendix}

% references
\printbibliography


\end{document}