\documentclass[main.tex]{subfiles}

\begin{document}
\section{Methods}

\subsection{Notation}

This section introduces the notation and definitions used in the present paper, focusing on discrete stationary time series.

\subsubsection*{Autocorrelation Function for Stationary Time Series}

For a discrete, stationary time series, $\{X_t, t\in\mathbb{Z}\}$, the autocorrelation function (ACF) at lag $k$ is defined as:

\begin{align*}
    \rho(k) = \frac{\mathbb{E}[(X_{t+k} - \mu_X)(X_t - \mu_X)]}{\sigma_X^2},
\end{align*}

where $\mu_X$ is the mean and $\sigma^2_X$ is the variance of the times series. When the process is stationary, $X_t$ is dependent and identically distributed, both the mean and variance are constant over time $t$. (comment on stationary and ACF decay)\\

\subsubsection*{Sample Autocorrelation Function}

The ACF of a finite sample $\{X_1, X_2, ..., X_n\}$ from a stationary time series, at lag $k$ ($0 \le k < n$), is estimated by:
\begin{align*}
    \hat \rho(k) = \frac{\sum_{t=1}^{n-k}(X_{t+k} - \bar X)(X_t - \bar X)}{\sum_{t=1}^n(X_t - \bar X)^2},
\end{align*}

where $\bar X$ represents the sample mean of the time series.

\subsection{Estimating Timescales}
Different methods have been used to estimate timescale parameters from neural signals, both using time and frequency domain models.\\

\subsubsection*{Time Domain}
\paragraph{(a) Decay rate of the sample ACF, estimated by nonlinear least-squares (NLS)}
\begin{align*}
    \hat \tau = \underset{{\tau}}{\text{min}} \sum_{k=0}^{n-1}[\hat \rho(k) - e^{-\frac{k}{\tau}}]^2,
\end{align*}

where $e^{-\frac{k}{\tau}}$ is an exponential function with decay rate $\tau$. 

\paragraph{(b) Best AR(1) model projection, estimated by ordinary least-squares (OLS)}
\begin{align*}
    \hat \phi &= \underset{\phi}{\text{min}} \ \mathbb{E}[(X_t - \phi X_{t-1})^2]\\
    \hat \tau &= -\frac{1}{\text{log}(\hat \phi)}
\end{align*}

\subsubsection*{Frequency Domain}
\paragraph{(c) PSD Knee Frequency, estimated by nonlinear least-squares (NLS)}
\begin{align*}
    ...
\end{align*}
Procedure that attempts to fit to the aperiodic aspects of the spectrum. scale-free ($\frac{1}{f}$) structure of spectral brain dynamics (more energy in low frequencies).

\subsection{Simulations}  
Realizations of stationary autoregressive time series were simulated to match...\\
filter which is convolved with a one-dimensional white noise signal...

\subsection{Evaluation}

\subsection{Applications}
\end{document}