\documentclass[latex/main.tex]{subfiles}

\begin{document}
\section{Conclusions}
This study introduces statistical methods for mapping fMRI timescales. We detail the large-sample properties of time- and autocorrelation-domain methods, showing both give estimates that converge, but to different values. This difference arises from the distinct definitions of the timescale parameter inherent to each method. We also demonstrate that both estimators yield consistent standard errors under broad conditions, allowing reliable inference and hypothesis testing. This addresses a major limitation in neural timescale studies that typically report only point estimates without uncertainty measures.\\

Simulations confirm that both methods provide valid timescale estimates and standard errors, assessing estimation accuracy. Specifically, the time-domain linear method \eqref{eq:ar1} is more computationally efficient and stable for small timescales, while the autocorrelation-domain nonlinear method \eqref{eq:nlm} can capture longer-range autocorrelations at the expense of inaccuracy. Newey-West corrected standard errors reduce bias across methods, enhancing inference reliability \citep{newey_simple_1987}. This is evident in AR(2) and empirical rfMRI simulations when autocorrelation decay is not strictly exponential. Applied to HCP rfMRI data, both methods yield timescale and t-statistic maps consistent with known functional hierarchies \citep{van_essen_wu-minn_2013},  as other studies similarly show larger timescales in associative versus sensory cortices \citep{raut_hierarchical_2020, shafiei_topographic_2020, lurie_cortical_2024, mitra_lag_2014, kaneoke_variance_2012, wengler_distinct_2020, shinn_functional_2023, manea_intrinsic_2022, ito_cortical_2020, muller_core_2020}. More detailed interpretation of the mechanistic interpretation of timescales is reserved for future work.\\

Although rfMRI provides high spatial resolution compared to EEG, MEG, or ECoG, its sparse temporal sampling can introduce finite sample variability. Asymptotic properties (see \nameref{sec:estimator-properties}) assume large samples, but in practice, low-frequency sampling of strongly dependent hemodynamic processes can yield too small an effective sample size \citep{afyouni_effective_2019, kaneoke_variance_2012}. Moreover, the mixed metabolic and neuronal origins of the hemodynamic signal complicate mechanistic interpretations \citep{raut_hierarchical_2020, he_scale-free_2011}. Methodologically, deviations from stationarity and ergodicity can still affect reliability, despite our methods handling common forms of misspecification \citep[Chapter~14.7]{hansen_econometrics_2022}. Extending model definitions to account for nonstationarity, where autocorrelations are time-dependent, might provide more accurate maps, especially in task-based or dynamic fMRI paradigms \citep{he_scale-free_2011}. Additionally, adding standard errors to the frequency-domain approach to timescale estimation would allow for the direct modeling of oscillations, which is important when working with electrophysiological recordings of the brain \citep{donoghue_parameterizing_2020, gao_neuronal_2020}.\\

In conclusion, we introduce robust rfMRI estimators for timescales and standard errors, enabling rigorous statistical comparisons across regions, conditions, and subjects. This advances the accuracy and interpretability of neural timescale maps. Our methods move beyond point estimates by incorporating variability for inference and testing. The work lays the methodological foundation for future research on the role of timescales in brain structure and function.
\end{document}