\documentclass[docs/main.tex]{subfiles}

\begin{document}
\section{Conclusions}
This study introduces statistical methods for mapping fMRI timescales. We detail the large-sample properties of time- and autocorrelation-domain methods, showing both give estimates that converge, but to different values. This difference arises from the distinct definitions of the timescale parameter inherent to each method. We also demonstrate that both estimators yield consistent standard errors under broad conditions, allowing reliable inference and hypothesis testing. This addresses a major limitation in neural timescale studies that typically report only point estimates without uncertainty measures.\\

Simulation results highlight differences in finite sample bias and variance between the time- and autocorrelation-domain methods. All timescale estimators show largely unbiased results, with the linear least squares (LLS) method performing as well as or better than nonlinear least squares (NLS) in terms of relative root mean square error (rRMSE). Standard error estimates exhibit pronounced differences between naive and Newey-West corrected methods, particularly under misspecified settings such as AR(2) processes and realistic rfMRI data. Notably, standard errors cannot be directly estimated in the autocorrelation domain from sample autocorrelation functions (ACFs) of time-series data, necessitating a hybrid approach as detailed in \nameref{sec:stderr-autocorrelation/time-domain_}. Furthermore, time-domain methods demonstrate comparable or superior performance to autocorrelation-domain methods across all simulation scenarios. Specifically, the time-domain linear method \eqref{eq:ar1} offers greater computational efficiency and estimation stability for smaller timescales, whereas the autocorrelation-domain nonlinear method \eqref{eq:nlm} accommodates longer-range autocorrelations at the cost of reduced accuracy. The application of Newey-West corrected standard errors enhances inference reliability by reducing bias across both methods \citep{newey_simple_1987}, particularly evident in scenarios where autocorrelation decay is not exponential.\\

Applied to HCP rfMRI data, both methods yield timescale and t-statistic maps consistent with known functional hierarchies \citep{van_essen_wu-minn_2013}. These results align with prior work demonstrating larger timescales in associative versus sensory cortices \citep{raut_hierarchical_2020, shafiei_topographic_2020, lurie_cortical_2024, mitra_lag_2014, kaneoke_variance_2012, wengler_distinct_2020, shinn_functional_2023, manea_intrinsic_2022, ito_cortical_2020, muller_core_2020}. The spatial patterns reinforce the hypothesis that hierarchical organization governs temporal processing across cortical regions. However, mechanistic interpretations of these timescales -- particularly their relationship to underlying neurophysiological processes -- remain to be fully explored in future work.\\

Although rfMRI provides high spatial resolution compared to EEG, MEG, or ECoG, its sparse temporal sampling can introduce finite sample variability. Asymptotic properties (see \nameref{sec:estimator-properties}) assume large samples, but in practice, low-frequency sampling of strongly dependent hemodynamic processes can yield too small an effective sample size \citep{afyouni_effective_2019, kaneoke_variance_2012}. Moreover, the mixed metabolic and neuronal origins of the hemodynamic signal complicate mechanistic interpretations \citep{raut_hierarchical_2020, he_scale-free_2011}. Methodologically, deviations from stationarity and mixing can still affect reliability, despite our methods handling common forms of misspecification \citep[Chapter~14.7]{hansen_econometrics_2022}. Extending model definitions to account for nonstationarity, where autocorrelations are time-dependent, might provide more accurate maps, especially in task-based or dynamic fMRI paradigms \citep{he_scale-free_2011}. Additionally, adding standard errors to the frequency-domain approach to timescale estimation would allow for the direct modeling of oscillations, which is important when working with electrophysiological recordings of the brain \citep{donoghue_parameterizing_2020, gao_neuronal_2020}.\\

In conclusion, we introduce robust rfMRI estimators for timescales and standard errors, enabling rigorous statistical comparisons across regions, conditions, and subjects. This advances the accuracy and interpretability of neural timescale maps. Our methods move beyond point estimates by incorporating variability for inference and testing. The work lays the methodological foundation for future research on the role of timescales in brain structure and function.
\end{document}
