\documentclass[docs/main.tex]{subfiles}

\begin{document}
\section{Introduction}

\subsection{fMRI Timescale Maps}
Neural processes span multiple timescales, from millisecond synaptic events to slower activity coordinating distributed brain networks \citep{buzsaki_large-scale_2004}. Multimodal evidence links these differences in timescales to intrinsic brain organization that reveal how regions integrate and process information over time. Timescale maps of the brain align with functional hierarchy -- sensory areas that process rapidly changing stimuli show faster timescales than association areas involved in cognitive processes that unfold over longer durations \citep{raut_hierarchical_2020, gao_neuronal_2020, hasson_hierarchy_2008, murray_hierarchy_2014, stephens_place_2013}. This hierarchy is also associated with anatomical organization including myelination levels and gene expression patterns, as shown in studies using human electrophysiology, MEG, and gene expression profiling \citep{gao_neuronal_2020, shafiei_neurophysiological_2023}.\\

Computational modeling by \citet{li_hierarchical_2022} suggests that hierarchical timescales emerge from: (1) brain-wide gradients in synaptic excitation strength, (2) electrophysiological differences between excitatory and inhibitory neurons, and (3) balance between distant excitatory and local inhibitory inputs. In addition to these intrinsic mechanisms, there is growing evidence that neuronal timescales are dynamic and modulated by experimental manipulations or behavioral demands. For example, pharmacological agents like propofol and serotonergic drugs alter intrinsic timescales, affecting the temporal integration of information in the brain \citep{huang_timescales_2018, shinn_functional_2023}. Timescale changes have also been observed during development, sleep deprivation, wakefulness, neuropsychiatric disorders (autism and schizophrenia), and naturalistic behaviors \citep{martin-burgos_development_2024, meisel_decline_2017, watanabe_atypical_2019, wengler_distinct_2020, manea_neural_2024}. These findings demonstrate that timescales are broadly relevant to both structural and functional properties of the brain.\\

Seminal research on timescales has primarily used invasive electrophysiology in non-human animals \citep{murray_hierarchy_2014, cirillo_neural_2018, nougaret_intrinsic_2021, manea_intrinsic_2022, spitmaan_multiple_2020, trepka_training-dependent_2024}. While these methods provide high temporal resolution for studying neural activity at the single-neuron level, they are limited by sparse spatial sampling. Investigating the large-scale spatial organization of timescale maps requires non-invasive methods like resting-state functional MRI (rfMRI), which measures spontaneous fluctuations in the blood oxygen level-dependent (BOLD) signal. rfMRI provides full-brain coverage of hemodynamic processes at frequencies below 0.1 Hz \citep{raut_hierarchical_2020, he_scale-free_2011}, offering dense spatial sampling compared to techniques like EEG or MEG. Although the BOLD signal does not directly measure neural activity, it reflects hemodynamic changes associated with underlying electrophysiological signals \citep{logothetis_what_2008}, making it a valuable tool for investigating high-resolution cortical timescale maps. Studies have shown that rfMRI-derived timescale maps align spatially with those from other imaging modalities across human and animal models \citep{raut_hierarchical_2020, shafiei_topographic_2020, lurie_cortical_2024}. The present study will focus on rfMRI from the Human Connectome Project dataset \citep{van_essen_wu-minn_2013}.

\subsection{Current methods}
Timescales are generally estimated using three main methods in the (1) time-domain, (2) autocorrelation-domain, or (3) frequency-domain. The most common is the autocorrelation domain, where timescales are defined by fitting an exponential decay model to the sample autocorrelation function (ACF) \citep{rossi-pool_invariant_2021, cirillo_neural_2018, ito_cortical_2020, runyan_distinct_2017, zeraati_flexible_2022, nougaret_intrinsic_2021, wasmuht_intrinsic_2018, muller_core_2020, maisson_choice-relevant_2021, li_hierarchical_2022, shafiei_topographic_2020}. Similar approaches compute timescales directly from the sample ACF as the sum of positive autocorrelations \citep{wengler_distinct_2020, manea_intrinsic_2022, watanabe_atypical_2019}, or by identifying where the sample ACF crosses a specified threshold \citep{wengler_distinct_2020, zilio_are_2021}. Alternatively, the time-domain method uses a first-order autoregressive (AR1) model to estimate timescales directly from time-series data \citep{kaneoke_variance_2012, meisel_decline_2017, huang_timescales_2018, lurie_cortical_2024, shinn_functional_2023, shafiei_topographic_2020, spitmaan_multiple_2020, trepka_training-dependent_2024}, and has shown better test-retest reliability than autocorrelation-domain methods for rfMRI \citep{huang_timescales_2018}. Finally, frequency-domain methods use the sample power-spectral density (PSD) to estimate and remove neural oscillations, as timescales are properties of the aperiodic signal \citep{donoghue_parameterizing_2020, gao_neuronal_2020, manea_neural_2024, zeraati_flexible_2022, fallon_timescales_2020}. Since previous research shows that rfMRI is predominantly aperiodic with scale-free spectral properties \citep{he_temporal_2010, he_scale-free_2011}, the present paper focuses only on the time- and autocorrelation-domain methods.

\subsection{Problem Statement and Proposed Solution}
A key challenge in applied timescale research is the lack of standardized model definitions, where diverse approaches have led to inconsistent findings across studies \citep{zeraati_flexible_2022, fallon_timescales_2020, shafiei_topographic_2020}. Many parameterization methods rely on restrictive assumptions, such as exponential autocorrelation decay, which may bias timescale estimates and (more often) their standard errors \citep{woolrich_temporal_2001, zeraati_flexible_2022, raut_time_2019}. Additionally, the distributional properties of these methods are often ignored, resulting in studies reporting only point estimates without quantifying uncertainty, which limits statistical inference and hypothesis testing \citep{newey_simple_1987, white_nonlinear_1984}.\\

To address these issues, this paper formalizes and evaluates two commonly applied timescale methods: the time-domain fit of an autoregressive (AR1) model and the autocorrelation-domain fit of an exponential decay model in rfMRI. The goal is to estimate accurate timescale maps that enable robust statistical testing and inference across brain regions. This work offers the following contributions: (1) The assumptions are generalized to include all stationary and mixing processes, not only those with exponential autocorrelation decay. (2) Robust standard errors account for the inevitable mismatch between the data-generating process and fitted model, enabling valid inference despite model misspecification. (3) Theoretical properties demonstrate that both time- and autocorrelation-domain estimators converge to different values due to their distinct definitions, and are consistent and asymptotically normal. (4) Simulations confirm that both methods yield unbiased estimates across autoregressive and realistic settings, with standard errors that are robust to non-exponential autocorrelation decay. (5) Empirical analysis of rfMRI from the Human Connectome Project shows that both approaches yield similar t-ratio maps (timescales relative to their standard errors), revealing a hierarchical organization of timescales across the cortex. While this hierarchy aligns with prior point estimate maps, our approach improves interpretability by accounting for variability, ensuring that observed patterns are not artifacts of sampling noise or model misspecification. (6) Comparative analyses show that the time-domain method performs as well as, and often better than, the autocorrelation-domain method, while maintaining greater computational efficiency for high-dimensional fMRI data analysis.\\

The proposed methods address important limitations in neural timescale research by providing rigorous statistical methods that move beyond point estimates to incorporate uncertainty quantification. Through formal definitions, theoretical validation, and extensive testing across simulations and empirical data, we demonstrate that both time- and autocorrelation-domain estimators yield consistent standard errors under broad conditions, enabling reliable inference and hypothesis testing. This work establishes a methodological foundation for future research investigating the functional and structural organization of timescales in the brain.


\end{document}
